

\documentclass[a4j]{jarticle}

\usepackage{listings}

\lstset{%
  language={C},
  basicstyle={\small},%
  identifierstyle={\small},%
  commentstyle={\small\itshape},%
  keywordstyle={\small\bfseries},%
  ndkeywordstyle={\small},%
  stringstyle={\small\ttfamily},
  frame={tb},
  breaklines=true,
  columns=[l]{fullflexible},%
  numbers=left,%
  xrightmargin=0zw,%
  xleftmargin=3zw,%
  numberstyle={\scriptsize},%
  stepnumber=1,
  numbersep=1zw,%
  lineskip=-0.5ex%
}

\title{プログラミング演習\\中間レポート}
\author{\\学籍番号:09425566\\氏名:戸塚佑太}
\date{出題日:2014/05/19\\提出日:2014/05/26\\締切り日:2014/05/26\\}

\begin{document}
\maketitle

\newpage



%
%	SECTION 1
%

\section{概要}

このレポートでは,標準入力からカンマ区切りのCSV形式のファイル,またはCSVデータを入力し,それら1行ずつ読み込み,区切りごとにid,name,birth,addr,commentの5つの項目に分けて格納し,表示するプログラムを作成する途中過程を示すものである.

\begin{enumerate}
\item 格納するデータを構造体として表現.指定されたデータ構造は以下の通りである.

\begin{center}
\begin{tabular}{|c|c|c|c|c|}\hline
\centering
ID&学校名&設立年月日&所在地&備考データ\\\hline
32bit整数&70bytes&struct date&70bytes&任意長\\\hline
\end{tabular}
\end{center}

この構造体を配列として10000件のデータを格納できるように宣言する.

\item 標準入力からの入力をCSV形式として読み込み,上記に指定された構造体の配列に格納する.SCVの形式は次の通り.

{\baselineskip 3mm
\begin{verbatim}
 0,Takahashi Kazuyuki,1977-04-27,Saitama,Fukuoka Softbank Howks
 1,Yuta Totsuka,1993-04-24,Okayama,Kurashiki
 2,Kubo Shota,1993-04-16,Ehime,Matuyamakita
 3,Oigawa Satoshi,1993-04-18,Shimane,Matueminami
 .
 .
\end{verbatim}
}

\item %から始まる文はCSV入力ではなくコマンドとみなして処理を行う.今回は%Q,%C,%Pコマンドのみ実装し,それぞれのコマンドは次の動作を行うよう実装する.

\begin{center}
\begin{tabular}{|c|c|c|}\hline
コマンド&意味&備考\\ \hline
%Q&終了(Quit)&\\ \hline
%C&登録件数の表示(Check)&\\ \hline
%P n&先頭からn件表示&n=0:全件表示,n$<$0:後ろから-n件表示\\ \hline
\end{tabular}
\end{center}

\end{enumerate}


%
%	SECTION 2
%

\section{プログラムの作成方針}

今回のプログラムは大きなプログラムとなるので,いくつかの処理に分けて関数を作成する.
処理の概要は以下の通りに定め,下記でそれぞれについて解説する.
\begin{itemize}
\item[(1)]格納を行う構造体の宣言部
\item[(2)]標準入力からの文章を1行読み込む
\item[(3)]標準入力データがCSVの場合の処理
\item[(4)]標準入力データがコマンドの場合の処理
\end{itemize}

まず,(1)格納を行う構造体の宣言部 については,概要で示した通りにデータを格納できるよう宣言する.

{\baselineskip 3mm
\begin{verbatim}
struct date{
  int y;
  int m;
  int d;
};

struct profile{
  int id;
  char name[MAX_STR_LEN+1];
  struct date birth;
  char home[MAX_STR_LEN+1];
  char *comment;
};

struct profile profile_data_store[MAX_PROFILES];
\end{verbatim}
}


{\baselineskip 3mm
\begin{verbatim}

(2)標準入力からの文章を1行読み込む は主にget_line,subst,perse_lineの部分で処理を行っている.標準入力されたデータをchar *lineで1行分読み込み,1文字目が%であれば2文字目以降のコマンドと引数を別関数の引数とし,各コマンドに応じた処理を行う.また,1文字目が%でない場合はこの1行をCSV形式の文とみなし,カンマ ',' を区切りとして5つの文字列として分割する.

(3)標準入力データがCSVの場合の処理 はnew_profile,new_date,splitの部分で処理を行っている.標準入力されたデータがCSVデータだった場合,1行毎に文字列として分割し,これらをnew_profileに送り,項目毎に適切な方に変換し,それぞれ構造体のメンバに代入する.文字列の場合はそのまま代入を行うためにstrncpy,数値の場合はatoiを使い変数変換を行い代入・格納する.設立年月日の部分(2013-6-6)の文字列もnew_dateに送り,'-'を区切りとして同様に文字列として分割し,数値変換を行ってから変数に格納する.
また,分割して送られてきた文字列はstrncpyを使用し,メモリ間のコピーを行わなければならないことに注意しなければならない.

(4)標準入力データがコマンドの場合の処理 は各コマンドの実現部分であり,プログラムの終了,登録件数・登録項目の表示を行う部分である.プログラムの終了はexit(0)を使用することにより,コマンド入力後に処理が停止する.登録件数はprintfで表示する.登録項目の表示は3文字目以降の引数の件数分(n件)をそれぞれ場合分けしてprintfで表示させる.場合分けの方法は,概要の示している通りに行っている.また登録件数を越えた引数(|nitems|>n)が送られた場合はerrorが表示されるようになっている.

\end{verbatim}
}

%
%	SECTION 3
%

\section{プログラムリストおよび、その説明}

\begin{verbatim}
 完成したプログラムを末尾に添付する.このセクションでは,プログラムの主な構造について説明する.
まず,8-20行付近はstruct dataのデータ型の宣言部とそれを扱う関数の宣言部である.次に,subst,splitを26-56行付近で宣言している.substはstrの文字列中のc1をc2へと変換する.ここでは','を'\0'へと変換している.splitでは送られてきたstrの文字列中の区切りsepで分割し,substと同様に','へと'\0'変換し,分割したものをret[]に格納している.これらの文字列を示す複数からなる配列を返す.また"2013-06-06"のような日付を分けるために分割文字を'-'としてstruct_dateで同様の処理を行っている.
次に58-67,195-226付近のget_line,perse_line,mainでは標準入力され文章を1行ごと読み込み,解析し,データが%から始まっていればコマンド文字と引数をexec_commandに送る.そうでなければ一行をnew_profileに送る.
また73-123行付近のnew_profile,new_dateでは解析を行い,送られてきた一行を分割し,格納を行う.ここで,"2013/06/07"のように'-'で区切られず,間違った形式で入力された場合は処理されず,はじかれる.上記のsplitで分割した無事列配列を構造体の宣言部のデータ型に変換し,代入を行っている.文字列はstrncpy,数値はatoi関数を使用.これらをprofile_data_storeに格納している.profile_data_storeに格納できる件数は最大10000件となっている

\end{verbatim}

%
%	SECTION 4
%

\section{プログラムの使用例・テスト}

本プログラムは名簿データを管理するためのプログラムである.標準入力されたCSV形式のデータまたはファイル,%から始まるコマンドに応じた処理をし,処理結果を標準出力に表示する.入力形式については概要を参照.
まず、本プログラム(main.c)をgccによりコンパイルし,a.outという実行ファイルを作成する。test.csvというCSVファイルの読み込み(入力)を行う場合は、下のように./a.out < test.csvと入力する。

{\baselineskip 3mm
\begin{verbatim}
% gcc main.c
% ./a.out < test.csv
\end{verbatim}
}

test.csvは以下のようであった場合を想定する。

{\baselineskip 3mm
\begin{verbatim}

1,Takahashi Kazuyuki,1977-04-27,Saitama,Fukuoka Softbank Howks
2,Yuta Totsuka,1993-04-24,Okayama,Kurashiki
3,Kubo Shota,1993-04-16,Ehime,Matuyamakita
4,Oigawa Satoshi,1993-04-18,Shimane,Matueminami
%P 0
%P 2
%P -2
%P 5
%C
\end{verbatim}
}

このとき以下のように、ユーザがより読み取りやすいように出力を得ることができる.

{\baselineskip 3mm
\begin{verbatim}

(line1)Id    : 1
Name  : Takahashi Kazuyuki
Birth : 1977-04-27
Addr  : Saitama
Com.  : Fukuoka Softbank Howks

(line2)Id    : 2
Name  : Yuta Totsuka
Birth : 1993-04-24
Addr  : Okayama
Com.  : Kurashiki

(line3)Id    : 3
Name  : Kubo Shota
Birth : 1993-04-16
Addr  : Ehime
Com.  : Matuyamakita

(line4)Id    : 4
Name  : Oigawa Satoshi
Birth : 1993-04-18
Addr  : Shimane
Com.  : Matueminami

(line1)Id    : 1
Name  : Takahashi Kazuyuki
Birth : 1977-04-27
Addr  : Saitama
Com.  : Fukuoka Softbank Howks

(line2)Id    : 2
Name  : Yuta Totsuka
Birth : 1993-04-24
Addr  : Okayama
Com.  : Kurashiki

(line3)Id    : 3
Name  : Kubo Shota
Birth : 1993-04-16
Addr  : Ehime
Com.  : Matuyamakita

(line4)Id    : 4
Name  : Oigawa Satoshi
Birth : 1993-04-18
Addr  : Shimane
Com.  : Matueminami

登録件数を確認してください.

登録件数:4件

\end{verbatim}
}

入力中の”%P 2”, "%P 0", "%P -2"はそれぞれ"前から2件表示","全件表示","後ろから2件表示"する処理を呼び出すコマンドである.
%Cは登録件数の表示をする処理を呼び出すコマンドである.

%
%	SECTION 5
%

\section{プログラム作成における考察}

\begin{verbatim}
プログラムの作成過程での考察は,分割して返された文字列を代入する際に,strncpyを使うようにした.数値の代入をするためにはatoi関数を使い値を直接代入するようにした.またcmd_print関数内では初め,すべてのnの場合分けを行いループを考え,その中のすべてで表示させていたが,記述量も多くなり,効率的では無いと考えたために,printで表示させる部分だけを別関数で作成し,ループ内に返されるように変更した.

\end{verbatim}

%
%	SECTION 6
%

\section{得られた結果に関する,あるいは諮問に対する回答}

\begin{verbatim}
struct profile *newprofileのように構造体の宣言にポインタがついているものがある.これはポインタを付けることによって,格納し,蓄積させたデータのすべてを返すのではなく先頭アドレスだけを返している.構造体内のすべての数値,文字列を返すよりも,効率が上がると考えたためである.また今回のプログラムではn件の登録件数に対し,その件数を上回る件数の表示を行おうとすると,登録件数を確認するように促し,表示がされないようにしている.この場合に表示を行った場合に,多少分かりにくくなってしまうのでは無いかと考え,まず登録件数を確認するように促すようにした.また最大の登録件数を越えて,新たなデータを登録しようとしたさいに,perse_line内で条件文により,最大登録件数になってしまっていることを伝え,そこで処理を終えるようになっている.
\end{verbatim}

\newpage

%
%	SECTION 6
%

\section{作成したプログラムのソースコード}



\lstinputlisting[caption=listManager.c, label=listManager]{listManager.c}


\end{document}

